\begin{frame}[label=Cartesio]
    \frametitle{Cartesio}
    \begin{block}{Geometria Algebrica}
        Descartes, in Francia, intorno al 1630, introduce la \alert{geometria algebrica}
        che permette di definire e classificare la classe delle \textit{curve algebriche}
        in base al loro \textit{grado}.
        Una curva algebrica piana è l'insieme dei punti $(x,y)$ che soddifano l'equazione
        \begin{center}
            $p(x,y) = 0$, dove $p(x,y)$ è un polinomio
        \end{center}
    \end{block}
    \pause
    \begin{block}{Retta tangente a una curva}
        Il problema delle tangenti è, per Descartes, \begin{quote}
            il problema più utile e generale [...] in Geometria
        \end{quote}.
        La sua soluzione, pubblicata nel 1637 nella \textit{Géométrie} è di considerare 
        la circonferenza tangente alla curva in un punto dato $P_0=(x_0,y_0)$. Una volta trovata quest'ultima, il suo
        raggio per $P_0$ sarà normale alla curva, e quindi la tangente sarà
        perpendicolare al raggio.
    \end{block}
    \pause
    \begin{alertblock}{Abbandono}
        Il metodo comporta calcoli piuttosto complicati, anche nei casi più semplici.
        Si tratta di un metodo di \textit{geometria algebrica} e non, come in Fermat,
        di \textit{calcolo differenziale}. \cite{Bourbaki}
    \end{alertblock}

\end{frame}