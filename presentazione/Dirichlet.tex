\begin{frame}[label=Dirichlet]
  \frametitle{Dirichlet}
  \begin{block}{Continuità e integrabilità}    
  Nel 1829 Dirichlet scopre che vi sono funzioni che restano integrabili anche se hanno \textit{qualche} discontinuità.
  Questo pone la questione: fino a che punto una funzione può essere discontinua e ammettere tuttavia un integrale?
  \end{block}  
  \begin{block}{La funzione di Dirichlet}    
    Dirichlet propone un esempio di funzione che è \textit{troppo discontinua} per essere integrabile nel senso di Cauchy.
    
    \begin{equation}
      D(x) =
      \begin{cases*}
        1 & se x è razionale \\
        0 & se x è irrazionale
      \end{cases*}
    \end{equation}

    In qualsiasi intervallo della retta reale ci sono sia punti razionali che punti irrazionali. Quindi data una qualsiasi suddivisione
    $ 0 = x_0 < x_1 < ... > x_n = 1$ , si possono sempre segliere le somme di Cauchy \begin{center}
      $(x_1 - x_0)D(x_0) + (x_2 - x_1)D(x_1) + ... + (x_n - x_{n-1})D(x_{n-1})$
    \end{center}
    in modo che i valori $D(x_0)$,$D(x_1)$, ... , $D(x_{n-1})$ siano tutti uguali a 0 o tutti uguali a 1. Di conseguenza non esiste un 
    unico valore limite, per cui l'integrale di Cauchy di $D(x)$ \alert{non esiste}.
  \end{block}  
  
\end{frame}


