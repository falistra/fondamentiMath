\begin{frame}[label=Fermat-metodo]
    \frametitle{Il metodo che non sbaglia mai}
    \begin{block}{L'``\textit{adeguagliazione}''}
        Fermat, nel trattato \textit{Methodus ad disquirendam maximam et minimam} (1636) scrive:
        ``\textit{L'intera dottrina della determinazione dei massimi e minimi è
        fondata su due espressioni simboliche e su questa sola regola:}''
        \begin{enumerate}
        \item<1-> La prima espressione contiene il termine da massimizzare : sia $a$.
        \item<2-> La seconda espressione si ottenga sostituendo $a+e$ al posto di $a$.
        \item<3-> Si ``\alert{adeguaglino}''[adaequentur] le due espressioni.
        \item<4-> Dopo aver tolto i termini comuni, si divida per $e$
        \item<5-> Si eliminino le quantità contenenti $e$, e si eguaglino [aequatur] i termini restanti
        \item<6-> La soluzione di quest'ultima equazione da il valore di $a$.
        \end{enumerate}
    \end{block}
    \pause

    \begin{exampleblock}{Esampio: Rettangolo di area massima}
    Dividere un segmento di lunghezza $b$ in due segmenti che siano base e altezza 
    di un rettangolo di area massima.
    \begin{enumerate}
        \item Sia $a$ il primo segmento. Il secondo sarà $b-a$. Si tratta di massimizzare l'espressione $a(b-a)$ (1).
        \item Si prenda ora $a+e$ e la si sostituisca al posto di $a$ nell'espressione da massimizzare, ottenendo
    l'espressione $((a+e)(b-(a+e))) = ba - a^2 +be -2ae -e^2$ (2).
    \item Si ``\alert{adeguaglino}'' l'espressione (1) e la (2) ottenendo un'``\alert{adequazione}'' ($\approx$).
    \item Semplificando e dividendo per $e$ si ottiene $b \approx 2a + e$
    \item Eliminando i termini che contengono $e$ resta $b = 2a$ 
    \item $a$ è la metà di $b$, quindi i due segmenti sono uguali. Il rettangolo che massimizza l'area è un quadrato.    
    \end{enumerate}
    \end{exampleblock}    

\end{frame}