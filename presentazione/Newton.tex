\begin{frame}[label=Newton]
  \frametitle{Newton}

  \begin{block}{De methodis serierum et fluxionum. 1670-1672}
    \textit{Per illustrare l'arte analitica non rimane ora che affrontare alcuni problemi
      ad essa inerenti che emergono soprattutto a causa della natura delle curve [...]
      tali difficoltà possono essere ricondotte a due soli problemi, che vorrei presentare
      in relazione allo spazio percorso con un qualsiasi moto locale, sia esso accelerato
      o ritardato:}

    \begin{itemize}
      \item
            \textit{Data la lunghezza della traiettoria in maniera continua (cioè, in ogni istante),
              trovare la velocità del moto in ogni istante.}
      \item
            \textit{Data la velocità del moto in maniera continua, trovare la lunghezza della traiettoria
              descritta (cioè della distanza percorsa) in ogni istante.}
    \end{itemize}

    \cite[Giusti]{Giusti} Ogni quantità è variabile \alert{fluente} rispetto al tempo. Due variabili $x$ e $y$ sono correlate 
    dall'equazione data $P(x,y)=0$ che determina la curva. 
    Newton introduce due nuove grandezze $\dot{x}$ e $\dot{y}$, che sono le velocità istantanee
    o \alert{flussioni}; i loro rapporti determinano la tangente / velocità alla curva e si possono ricavare operando
    \alert{secondo opportune regole} su $P(x,y)$.
  \end{block}
  \begin{exampleblock}{Calcolo delle flussioni : la regola del prodotto}
    Per trovare la velocità del prodotto, Newton considera un tempuscolo (Infinitesimo) $o$, dopo il quale $x$ e $y$ saranno diventate 
    rispettivamente $x+o\dot{x}$ e $y+o\dot{y}$. Allora la velocità sarà
    \begin{center}
      $\frac{(x+o\dot{x})(y+o\dot{y}) - xy}{o} = \frac{o(\dot{x}y + x\dot{y}) + o^2\dot{x}\dot{y}}{o} = \dot{x}y + x\dot{y} + o\dot{x}\dot{y}$
    \end{center}
    e, dato che l'ultimo termine $o\dot{x}\dot{y}$ è un infinitesimo per via di $o$, resta che la velocità del prodotto,
    denotata con $\dot{xy}$, è:
    \begin{center}
        $\dot{xy} = \dot{x}y + x\dot{y}$
    \end{center}

  \end{exampleblock}




\end{frame}

\begin{frame}[label=Newton teorema]

  \begin{block}{Estensione del metodo}
    Nel \textit{De methodis} Newton estende la soluzione, nota a Torricelli e Barrow per la
    classe delle curve $y = x^n$, alla più ampia classe delle \textit{serie infinite di potenze}.
    Newton è in grado di ottenere gli sviluppi in serie di tutte le quantità variabili (modenamente ``funzioni'').
    Ecco come risorlve il problema della quadratura, riducendolo esenzialmente all'integrazione delle potenze.

    \textit{
      \begin{enumerate}
        \item Se $x^{\frac{m}{n}}$ sono le ordinate alzate ad algolo retto, allora l'area della figura 
        sarà $\frac{n}{n+m}x^{frac{m+n}{n}}$
        \item Se l'ordinata è costituita da due o più ordinate unite dai segni $+$ e $-$, anche l'area 
        sarà allora costituita da due o più aree congiunte insieme rispettivaemnte dai segni $+$ e $-$.
        \item Ridurre le frazioni, i radicali, le radici affette da esponente in serie convergenti, quando
        non è possibile trovare altrimenti la quadratura; e nel quadrare, secondo le regole prima e seconda, le 
        figure le cui ordinate sono i singoli termini della serie.       
      \end{enumerate}
    }
  \end{block}

  \begin{block}{TFC secondo Newton}
    Per ogni serie di potenze, l'operazione di differenziazione è
    inversa all'operazione di integrazione. Tuttavia occorre
    avere una serie di potenze in forma esplicita.
  \end{block}

\end{frame}