\begin{frame}[label=Newton]
  \frametitle{Newton}

  \begin{block}{De methodis serierum et fluxionum. 1670-1672}
    \textit{Per illustrare l'arte analitica non rimane ora che affrontare alcuni problemi
    ad essa inerenti che emergono soprattutto a causa della natura delle curve [...]
    tali difficoltà possono essere ricondotte a due soli problemi, che vorrei presentare 
    in relazione allo spazio percorso con un qualsiasi moto locale, sia esso accelerato
    o ritardato:}
  
    \begin{itemize}
      \item
      \textit{Data la lunghezza della traiettoria in maniera continua (cioè, in ogni istante),
        trovare la velocità del moto in ogni istante.}
      \item
      \textit{Data la velocità del moto in maniera continua, trovare la lunghezza della traiettoria
        descritta (cioè della distanza percorsa) in ogni istante.}
      \end{itemize}
  \end{block}

\end{frame}

\begin{frame}
  \frametitle{Il TFC secondo Newton}
  \begin{theorem}{Teorema}
    Per ogni serie di potenze, l'operazione di differenziazione è 
    inversa all'operazione di integrazione. 
  \end{theorem}
  \begin{block}
    Assumendo che sia sempre possibile
    differenziare e integrare termine a termine.
  \end{block}
\end{frame}

\begin{frame}[label=Newton teorema]
    \begin{block}{TFC secondo Newton}
      Per ogni serie di potenze, l'operazione di differenziazione è 
      inversa all'operazione di integrazione. Assumendo che sia sempre possibile
      differenziare e integrare termine a termine.
    \end{block}
    \pause
    Questo teorema è una conseguenza diretta del fatto che la differenziazione è
    l'operazione inversa dell'integrazione per ogni potenza $x^n$. Tuttavia occorre
    avere una serie di potenze in forma esplicita.
  
\end{frame}