\begin{frame}[label=Riemann]
  \frametitle{Riemann}
  \begin{block}{La funzione di Dirichlet modificata}    
    Nel 1854 Riemann presentò una esempio di funzione che ha discontinuità in ogni intervallo ma rimane comunque integrabile secondo Cauchy.
    L'esempio più semplice di funzione con queste caratteristiche è:    

    \begin{equation}
      d(x) =
      \begin{cases*}
        \frac{1}{q} & se x = $\frac{p}{q}$ dove $\frac{p}{q}$ è ridotta ai minimi termini \\
        0 & se è irrazionale
      \end{cases*}
    \end{equation}

    Calcoliamo l'integrale. Per ogni $q \in \mathbb{N}$ :
    \begin{itemize}
      \item $d(x) > \frac{1}{q}$ solo per un numero finito di punti. Racchiusi in intervalli sufficientemente piccoli, 
      il loro contributo alla somma di Cauchy diventa arbitrariamente basso ($ < \frac{1}{q} $)
      \item Nell'intervallo $x \in [0,1]$, il resto dell'integrale è sempre racchiuso in un numero di rettangoli 
      di altezza $\leq \frac{1}{q}$ e quindi di area totale $< \frac{1}{q}$
    \end{itemize}

    Sommando i due contributi, l'area totale $< \frac{2}{q}$ per ogni $q \in \mathbb{N}$ quando l'ampiezza dei sottointervalli tende a zero. Quindi
    \begin{center}
      $\int_{0}^{1} f(x) \,dx = 0$
    \end{center}
    \end{block}    
\end{frame}


