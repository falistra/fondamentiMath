\begin{frame}[label=Riemann]
  \frametitle{Riemann}
  \begin{block}{La funzione di Dirichlet modificata}    
    Nel 1854 Riemann presentò una esempio di funzione che ha discontinuità in ogni intervallo ma rimane comunque integrabile secondo Cauchy.
    L'esempio più semplice di funzione con queste caratteristiche è:    

    \begin{equation*}
      d(x) =
      \begin{cases*}
        \frac{1}{q} & se x = $\frac{p}{q}$ dove $\frac{p}{q}$ è ridotta ai minimi termini \\
        0 & se è irrazionale
      \end{cases*}
    \end{equation*}

    Calcoliamo l'integrale. Per ogni $q \in \mathbb{N}$ :
    \begin{itemize}
      \item $d(x) > \frac{1}{q}$ solo per un numero finito di punti. Racchiusi in intervalli sufficientemente piccoli, 
      il loro contributo alla somma di Cauchy diventa arbitrariamente basso ($ < \frac{1}{q} $)
      \item Nell'intervallo $x \in [0,1]$, il resto dell'integrale è sempre racchiuso in un numero di rettangoli 
      di altezza $\leq \frac{1}{q}$ e quindi di area totale $< \frac{1}{q}$
    \end{itemize}

    Sommando i due contributi, l'area totale $< \frac{2}{q}$ per ogni $q \in \mathbb{N}$ quando l'ampiezza dei sottointervalli tende a zero. Quindi
    \begin{center}
      $\int_{0}^{1} d(x) \,dx = 0$
    \end{center}
    \end{block}    
\end{frame}

\begin{frame}[label=L'integrale di Riemann]
  \frametitle{L'integrale di Riemann}
  \begin{block}{L'integrale di Riemann}    
    Dato un intervallo $[a,b]$ , \textit{suddivisione} è ogni \textit{sottoinsieme finito} $\sigma$ di $[a,b]$ che contiene gli estremi $a$ e $b$. \\

    Data una funzione $f : [a,b] \rightarrow \mathbb{R}$ \textit{limitata} e $ \mathcal{D} = \left\{x_0,...,x_n\right\} $ una suddivisione di $[a,b]$,\\
    chiamiamo rispettivamente \textit{somma inferiore} $s(\mathcal{D},f)$ e \textit{somma superiore} $S(\mathcal{D},f)$ di Riemann :
    \begin{center}
      \[s(\mathcal{D},f) = \sum_{k = 1}^n (x_k-x_{k-1})\inf_{(x_{k-1},x_k)} f\]
      \[S(\mathcal{D},f) = \sum_{k = 1}^n (x_k-x_{k-1})\sup_{(x_{k-1},x_k)} f\]
    \end{center}

    Una funzione $f : [a,b] \rightarrow \mathbb{R}$ \textit{limitata} è integrabile secondo Riemann se è unico il numero reale $I$ compreso 
    fra \textit{tutte} le somme inferiori di $f$ e tutte le somme superiori di $f$. Se $f$ è integrabile secondo Riemann, l'unico numero $I$ 
    nelle condizioni dette è chiamato integrale di Riemann di $f$

    \begin{center}
      \[ I := \int_{(a,b)} f(x) \,dx := \sup_\mathcal{D} s(\mathcal{D},f)  =  \inf_\mathcal{D} S(\mathcal{D},f) \]
    \end{center}


    \end{block}    




\end{frame}
