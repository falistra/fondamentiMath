\begin{frame}[label=Robinson]
  \frametitle{Robinson}
  \begin{block}{Quanto è comodo l'infinitesimo}
    Negli anni 60 del secolo XX, A.Robinson ri-propone la nozione di infinitesimo: un numero che è infinitamente piccolo
    e tuttavia maggiore di zero. Fermat prima e poi Newton e Leibniz lo avevano usato con estrema disinvoltura, benchè palesemente
    contradditorio. Ma Berkeley nel suo trattato \textit{The Analyst}, pubblicato nel 1734 scrive:
      \begin{quotation}
        Essi non sono nè quantità finite, nè quantità infinitamente piccole e neppure nulla. Non dobbiamo forse
        chiamarli i fantasmi di quantità defunte?
      \end{quotation}
      Per soddisfare le esigenze della logica, il Calcolo viene riformulato nel XIX secolo da Weierstass
      con l'argomentazione $\epsilon - \delta$, senza infinitesimi. 
      Ma proprio grazie ricorrendo alla logica che Robinson ha riportato di nuovo in vita l'infinitesimo.
    \end{block}
\end{frame}

\begin{frame}[label=Robinson]
  \frametitle{La pietra cade $s = 4,9t^2$ \cite{Davis}}
\begin{columns}[T] % align columns
  \begin{column}{.48\textwidth}
    WEIERSTASS
  \color{red}\rule{\linewidth}{4pt}
  
  Sia $t=1$ e $t' = 1 + \Delta t$ \\
  $\Delta t$ è un numero reale positivo.\\
  $s' = 4.9 + 9.8 \Delta t + 4.9(\Delta t)^2$ \\
  \setlength{\parskip}{2em}

  $\Delta s = s' - s = 9.8 \Delta t + 4.9(\Delta t)^2$ \\
  \setlength{\parskip}{2em}

  $\frac{\Delta s}{\Delta t} = 9.8 + 4.9 \Delta t$ \\
  \setlength{\parskip}{3em}

  Assegnato un qualsiasi numero reale positivo $\epsilon$, arbitrariamente piccolo, scegliamo
  $\delta = \frac{\epsilon}{4,9}$. \\
  \setlength{\parskip}{4em}
  
  Allora per tutti i $\Delta t < \delta$ \\
  $\frac{\Delta s}{\Delta t} - 9.8 = 4.9 \Delta t < 4.9 \delta  = 4.9 \frac{\epsilon}{4.9} = \epsilon$\\
  \setlength{\parskip}{4em}

  Quindi: Velocità istantanea = $ \lim_{\Delta t \to 0} \frac{\Delta s}{\Delta t} = 9.8 $ 
  
  \end{column}%
  \hfill%
  \begin{column}{.48\textwidth}
    ROBINSON
  \color{blue}\rule{\linewidth}{4pt}
  
  Sia $t=1$ e $t' = 1 + dt$ \\
  $dt$ è un numero positivo infinitesimo.\\
  $s' = 4.9 + 9.8 dt + 4.9(dt)^2$ \\
  \setlength{\parskip}{2em}

  $\Delta s = s' - s = 9.8 dt + 4.9(dt)^2$ \\
  \setlength{\parskip}{2em}

  $\frac{ds}{dt} = 9.8 + 4.9 dt$ \\
  \setlength{\parskip}{3em}

  Poichè $dt$ è un infinitesimo, lo è anche $4.9dt$. 9.8 è un numero reale standard.
  \setlength{\parskip}{4em}
  
    Quindi: Velocità istantanea = parte standard di $\frac{ds}{dt} = 9.8$ 
  
  \end{column}%
  
  \end{columns}
\end{frame}