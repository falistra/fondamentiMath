% $Header$

\documentclass[8pt]{beamer}
\geometry{paperwidth=160mm,paperheight=105mm}


\mode<presentation>
{
  \usetheme{Warsaw}
  % or ...

  \setbeamercovered{transparent}
  % or whatever (possibly just delete it)
}

\usepackage{multimedia}

\usepackage[english]{babel}
% or whatever

\usepackage[utf8]{inputenc}
% or whatever

\usepackage[T1]{fontenc}
% Or whatever. Note that the encoding and the font should match. If T1
% does not look nice, try deleting the line with the fontenc.

\usepackage{graphicx}
\graphicspath{{immagini/}}

\usepackage{wrapfig}

\title[Il Teorema Fondamentale del Calcolo] % (optional, use only with long paper titles)
{Il Teorema Fondamentale del Calcolo}

\subtitle
{È davvero possibile che la strada più breve per la verità passi attraverso qualcosa di falso ?} % (optional)

\author[F.~Zanasi, Matricola 48359] % (optional, use only with lots of authors)
{F.~Zanasi\inst{1} }
% - Use the \inst{?} command only if the authors have different
%   affiliation.

\institute[Universities of Somewhere and Elsewhere] % (optional, but mostly needed)
{
  \inst{1}%
  Corso di Laurea in Didattica e Comunicazione delle Scienze\\
  Università di Modena e Reggio Emilia
}
% - Use the \inst command only if there are several affiliations.
% - Keep it simple, no one is interested in your street address.

\date[Short Occasion] % (optional)
{26 Novembre 2021 / Corso di Fondamenti di Matematica}

\subject{Talks}
% This is only inserted into the PDF information catalog. Can be left
% out. 



% If you have a file called "university-logo-filename.xxx", where xxx
% is a graphic format that can be processed by latex or pdflatex,
% resp., then you can add a logo as follows:

% \pgfdeclareimage[height=0.5cm]{university-logo}{university-logo-filename}
% \logo{\pgfuseimage{university-logo}}



% Delete this, if you do not want the table of contents to pop up at
% the beginning of each subsection:
%\AtBeginSubsection[]
%{
%  \begin{frame}<beamer>{Outline}
%    \tableofcontents[currentsection,currentsubsection]
%  \end{frame}
%}


% If you wish to uncover everything in a step-wise fashion, uncomment
% the following command: 

%\beamerdefaultoverlayspecification{<+->}

\begin{document}

\begin{frame}
  \titlepage
\end{frame}

\begin{frame}{Sommario}
  \tableofcontents
  % You might wish to add the option [pausesections]
\end{frame}

\section{Introduzione}
\subsection{Senza formule}
\begin{frame}{Senza formule}
  \begin{center}
    \includegraphics[scale=.2]{Z-158.png}
    \includegraphics[scale=.2]{Z-170.png}
    \includegraphics[scale=.2]{Z-242.png}
    \includegraphics[scale=.2]{Z-287.png}
    \includegraphics[scale=.2]{Z-323.png}
    \includegraphics[scale=.2]{Z-418.png}
    \movie{\includegraphics[width=0.5\textwidth]{Z-510.png}}{Fundamental_theorem_of_calculus__animation_.avi}
  \end{center}
\end{frame}

\subsection{Storia}
\begin{frame}{Storia}
  \begin{block}{Origine}
    Il teorema nacque nel XVII secolo, quando si scoprì che i processi per determinare
    \begin{itemize}
      \item
            la tangente a una curva.
      \item
            l'area racchiusa da una curva.
    \end{itemize}
    erano l'uno l'inverso dell'altro.
  \end{block}

  \pause
  \begin{block}{Sviluppo}
    Nelle prime formulazioni 
    il teorema stabiliva che la differenziazione e l'integrazione di funzioni
    rappresentano operazioni inverse.
    In seguito la formulazione del teorema continuò a trasformarsi e parallelamente
    si precisarono e ampliarono le nozioni di \textit{differenziazione}, \textit{integrazione} e \textit{funzione}.
  \end{block}
\end{frame}

\subsection{Infinitesimi}
\begin{frame}{Infinitesimi}
  \begin{alertblock}{Infinitesimi}
    Una delle caratteristiche piú salienti della storia di questo teorema è il ruolo problematico
    e in qualche modo irritante delle grandezze infinitesimali, un concetto che sembrò per lungo tempo
    assurdo, per quanto indispensabile.
  \end{alertblock}
  \pause
  \begin{block}{Definizione}
    \alert{Infinitesimo} in Matematica, si dice di quantità variabile che,
    in opportune condizioni, ha per limite lo zero.
    La definizione del concetto di i. è dovuta ad A.-L. Cauchy (1821).
    Secondo tale definizione,
    l’i. non va inteso in senso di i. attuale (quantità infinitamente piccola,
    evanescente, e tuttavia diversa dallo zero),
    ma nel senso di i. potenziale (quantità che tende ad annullarsi).
  \end{block}

  \pause
  \begin{block}{Ex malo, bonum}
    La ricerca di una soluzione al problema degli infinitesimi condusse non solo a chiarire il concetto
    di funzione, ma anche a precisare il concetto di \textit{numero}.
  \end{block}

  \pause
  \begin{block}{Importanza}
    Il teorema fondamentale del Calcolo, \alert{(TFC)} fu dunque all'origine
    della revisione della matematica, e ciò ne fa uno dei teoremi piú importanti
    della sua storia.
  \end{block}

\end{frame}

\section{Il teorema del moto}
\subsection{Oresme}
\begin{frame}[label=Oresme]
  \frametitle{Oresme}
    Nel 1361,il matematico Oresme rappresentò il moto con una serie
    di grafici in cui \alert{la velocità} dipendeva dal \alert{tempo}.
    \begin{center}
    \includegraphics[scale=.4]{Oresme.png}    
    \end{center}
    \pause
    Egli dedusse che la distanza percorsa da un corpo \textit{A} che si muove con 
    accelerazione costante è pari a quella di un corpo \textit{B} che si muove con 
    velocità costante pari alla media delle velocità iniziale e finale del corpo a.
    \pause
    \begin{block}{TFC secondo Oresme}
      Oresme assume che la distanza percorsa da un corpo qualsiasi è pari
      all' \alert{area} sottesa dal grafico velocità-tempo.
    \end{block}  
  
  \end{frame}
\subsection{Torricelli}
\begin{frame}[label=Torricelli]
  \frametitle{Torricelli}
    Dato il grafico \textit{distanza-tempo} di un punto che si muove, diciamo
    con velocità \textit{v} al tempo \textit{t}, 
    \begin{center}
        \includegraphics[scale=.4]{distanza-tempo.png}
    \end{center}
    il \alert{coefficiente angolare} misura l'inclinazione della tangente al tempo \textit{t}.
    La velocità è il coefficiente angolare della curva nel grafico distanza-tempo)
    \textit{(Torricelli 1640)}
  \begin{block}{TFC secondo Torricelli}
    \begin{itemize}
        \item La distanza è l'area della velocità (in relazione al tempo)
        \item La velocità è il coefficiente angolare della tangente alla distanza (in relazione al tempo)
    \end{itemize}
  \end{block}  
\end{frame}

\section{Tangenti e aree}
\subsection{Cartesio}
\begin{frame}[label=Cartesio]
    \frametitle{Cartesio}
    \begin{block}{Geometria Algebrica}
        Descartes, in Francia, intorno al 1630, introduce la \alert{geometria algebrica}
        che permette di definire e classificare la classe delle \textit{curve algebriche}
        in base al loro \textit{grado}.
        Una curva algebrica piana è l'insieme dei punti $(x,y)$ che soddifano l'equazione
        \begin{center}
            $p(x,y) = 0$, dove $p(x,y)$ è un polinomio
        \end{center}
    \end{block}
    \pause
    \begin{block}{Retta tangente a una curva}
        Il problema delle tangenti è, per Descartes, \begin{quote}
            il problema più utile e generale [...] in Geometria
        \end{quote}.
        La sua soluzione, pubblicata nel 1637 nella \textit{Géométrie} è di considerare 
        la circonferenza tangente alla curva in un punto dato $P_0=(x_0,y_0)$. Una volta trovata quest'ultima, il suo
        raggio per $P_0$ sarà normale alla curva, e quindi la tangente sarà
        perpendicolare al raggio.
    \end{block}
    \pause
    \begin{alertblock}{Abbandono}
        Il metodo comporta calcoli piuttosto complicati, anche nei casi più semplici.
        Si tratta di un metodo di \textit{geometria algebrica} e non di, come in Fermat,
        di \textit{calcolo differenziale}. \cite{Bourbaki}
    \end{alertblock}

\end{frame}
\subsection{Fermat}
\begin{frame}[label=Fermat]
    \frametitle{Fermat}
    \begin{block}{Introduzione dell'infinitesimo}
        \begin{wrapfigure}{R}{0.3\textwidth}
            \centering
            \includegraphics[width=0.25\textwidth]{Tangente-parabola.png}
        \end{wrapfigure}

        Fermat, intorno al 1630, ha già un suo metodo per trovare la tangente, 
        grazie ad un \textit{espediente algebrico}, 
        che divenne un concetto nuovo: \alert{l'infinitesimo}.
        \pause
        Cerchiamo il coefficiente angolare della retta tangente a una parabola
        $y = x^2$ nel punto $x=1$. Dato il punto $P_0=(1,1)$, che giace sulla curva 
        e ha ascissa $x=1$ si consideri 
        un punto \textit{infinitamente vicino ad esso}, che ha per ascissa $x=1+dx$
        (ove con $dx$ indichiamo appunto un \textit{infinitesimo}).
        L'ordinata sarà, secondo l'equazione data della parabola,
        $y= {(1+dx)}^2 = {1 + 2dx + {(dx)}^2}$.
        Il coefficiente angolare che congiunge questi due punti $P_0=(1,1)$ e 
        $P_1=(1+dx,{(1+dx)}^2)$ è dato dal rapporto fra le differenze fra le coordinate:
        \begin{center}
            $\frac{{(1+dx)}^2-1}{dx} = \frac{2dx+{(dx)}^2}{dx} = 2 + dx$
        \end{center}
        che è infinitamente vicino a 2. Sembra dunque ragionevole affermare che 
        il coefficiente angolare della tangente nel punto $P_0=(1,1)$ è $2$ e quindi 
        l'equazione della retta tangente in questo punto è
            $(y-1) = 2(x-1)$ , ovvero $y = 2x -1$
    \end{block}
\end{frame}
\subsection{Cavalieri}
\begin{frame}[label=Cavalieri-Torricelli]
    \frametitle{Cavalieri-Torricelli}
    \begin{block}{Tangente e area di $y = x^n$}
        Con il suo metodo, Fermat, trovò che il coefficiente angolare della curva $y = x^n$
        in $x=a$ è $na^{n-1}$. Bonaventura Cavalieri, nella sua \textit{Geometria indivisibilium} (1635)
        considerò l'area sottesa alla curva $y= x^n$ come la somma di una collezione di ``indivisibili'',
        e giunse a determinare che l'area sottesa alla curva $y= x^n$, delimitata dalle ascisse
        $x=0$ e $x=1$ per ogni valore di $n$ è $\frac{1}{n+1}$.
        \\
        Torricelli, nel 1640, considerò la curva $y = x^n$ come un grafico velocità-tempo,
        dove l'area rappresenta la distanza.
        Per il \alert{TFC}, versione ``Teorema fondamentale del moto'', la velocità è il 
        coefficiente angolare del grafico distanza-tempo. Quale curva ha coefficiente angolare
        $x^n$?  Torricelli mostrò che $y = \frac{x^{n+1}}{n+1}$ ha ha coefficiente angolare
        $x^n$. Quindi l'equazione del grafico distanza-tempo è 
        \begin{center}
            $y = \frac{x^{n+1}}{n+1}$
        \end{center}
        \begin{center}
        \includegraphics[scale=0.5]{Area-sotto-curve.png}
        \end{center}
    \end{block}
\end{frame}

\section{Prime versioni del TFC}
\subsection{Newton}
\begin{frame}[label=Newton]
  \frametitle{Newton}

  \begin{block}{De methodis serierum et fluxionum. 1670-1672}
    \textit{Per illustrare l'arte analitica non rimane ora che affrontare alcuni problemi
    ad essa inerenti che emergono soprattutto a causa della natura delle curve [...]
    tali difficoltà possono essere ricondotte a due soli problemi, che vorrei presentare 
    in relazione allo spazio percorso con un qualsiasi moto locale, sia esso accelerato
    o ritardato:}
  
    \begin{itemize}
      \item
      \textit{Data la lunghezza della traiettoria in maniera continua (cioè, in ogni istante),
        trovare la velocità del moto in ogni istante.}
      \item
      \textit{Data la velocità del moto in maniera continua, trovare la lunghezza della traiettoria
        descritta (cioè della distanza percorsa) in ogni istante.}
      \end{itemize}
  \end{block}

\end{frame}

\begin{frame}
  \frametitle{Il TFC secondo Newton}
  \begin{theorem}{Teorema}
    Per ogni serie di potenze, l'operazione di differenziazione è 
    inversa all'operazione di integrazione. 
  \end{theorem}
  \begin{block}
    Assumendo che sia sempre possibile
    differenziare e integrare termine a termine.
  \end{block}
\end{frame}

\begin{frame}[label=Newton teorema]
    \begin{block}{TFC secondo Newton}
      Per ogni serie di potenze, l'operazione di differenziazione è 
      inversa all'operazione di integrazione. Assumendo che sia sempre possibile
      differenziare e integrare termine a termine.
    \end{block}
    \pause
    Questo teorema è una conseguenza diretta del fatto che la differenziazione è
    l'operazione inversa dell'integrazione per ogni potenza $x^n$. Tuttavia occorre
    avere una serie di potenze in forma esplicita.
  
\end{frame}
\subsection{Leibniz}
\begin{frame}[label=Leibniz]
    \frametitle{Leibniz}
    \begin{block}{Nova Methodus}
    Nel 1684 Leibniz diede alle stampe l'opera \textit{Nova Methodus pro maximis e minimis}, 
    la prima pubblicazione sul \alert{calcolo differenziale}  
    inteso nell'accezione moderna: un metodo e un simbolismo generali
    per il calcolo delle tangenti alle curve.
    \end{block}
    \pause
    \begin{block}{Prima esposizione moderna del Calcolo Differenziale}
        Troviamo la notazione $\frac{dy}{dx}$, le regole di differenziazione e il concetto 
        di funzione (anzi la parola stessa). Leibniz introduce la notazione \textit{dx} 
        per denotare un incremento infinitesimo di x (la "d" sta per differenza).
    \end{block}
    \pause
    \begin{exampleblock}{Calcolo differenziale: la regola del prodotto}
        Per esempio se $y = uv$ dove u,v sono funzioni della $x$.
        L'incremento $dy$ diventa:
        \begin{center}
            \scalebox{1.5}{%
            $dy=(u+du)(v+dv)-uv = udv+vdu+ dudv$%
            }
        \end{center}
        e il coefficiente angolare $\frac{dy}{dx}$ della retta che passa 
        per il punto $(x,y)$ e il punto infinitamente vicino $(x+dx,y+dy)$
        si ottiene, a meno di un infinitesimo, semplicemente dividendo per $dx$:
        \begin{center}
            \scalebox{1.5}{%
            $\frac{dy}{dx} =  u\frac{dv}{dx} + v\frac{du}{dx} $%
            }
        \end{center}
        Il termine $dudv$ si elide in quanto e' un infinitesimo ``di ordine superiore.''
    \end{exampleblock}
\end{frame}

\begin{frame}
    \frametitle{L'integrale secondo Leibniz}
    \begin{block}{La definizione}
        Nel 1686 Leibniz da alle stampe la prima pubblicazione 
        sul Calcolo integrale. Introduce la notazione $\int$$ydx$ per indicare
        la funzione $y$ di $x$ , dove $\int$, una S allungata sta per "somma".
        Il termine seguente, $ydx$, indica l'area di un rettangolo infinitesimo 
        di altezza $y$ e base $dx$. 
        Quindi $\int$$ydx$ denota la somma di queste aree infinitesime: 
        l'area sottesa alla curva la cui altezza in $x$ è $y$.  
    \end{block}
    \begin{center}
        \includegraphics[scale=.6]{Area-somma-rettangoli.png}
    \end{center}
\end{frame}

\begin{frame}
    \frametitle{Il TFC secondo Leibniz}
    \begin{alertblock}{Domanda}
    \begin{center}
        \fontsize{15}{17.2}\selectfont
        Che cosa significa $d$$\int$$ydx$ ?
    \end{center}
    \end{alertblock}
    \begin{block}{Teorema}
        Poiché $d$ significa "\textit{incremento infinitesimo}" e $\int$
        significa "\textit{somma}", allora $d$$\int$$ydx$ significa
        "\textit{incremento Infinitesimo della somma (di infiniti $ydx$} \textit{)}",
        La risposta alla domanda precedente é sicuramente :
        \begin{center}
        \scalebox{2}{%
                    $d$$\int$$ydx = ydx$%
                    }
        \end{center}
        Quindi
        \begin{center}
            \scalebox{2}{%
            $\frac{d}{dx}$$\int$$ydx = y$%
            }
        \end{center}
        In parole: \textit{Se si integra una funzione $y$ e poi si 
        differenzia il risultato si ottiene di nuovo la funzione $y$
        }
    \end{block}

    \begin{block}{Fondamento del calcolo... nelle parole di Leibniz\cite{Mugnai}}
        Le differenze e le somme sono tra loro reciproche, vale a dire che la somma delle differenze
        della successione è il termine della successione, mentre la differenza delle somme della successione
        è lo stesso termine della successione: la prima affermazione la enuncio così. $\int \,dx = x$,
        la seconda così: $d \int x = x$ 
    \end{block}
\end{frame}


\begin{frame}
    \frametitle{Bibliografia}
    \begin{thebibliography}{Dijkstra, 1982}
        \bibitem[Salomaa, 1973]{Salomaa1973}
        A.~Salomaa.
        \newblock {\em Formal Languages}.
        \newblock Academic Press, 1973.
        \bibitem[Dijkstra, 1982]{Dijkstra1982}
        E.~Dijkstra.
        \newblock Smoothsort, an alternative for sorting in situ.
        \newblock {\em Science of Computer Programming}, 1(3):223--233, 1982.
    \end{thebibliography}
\end{frame}

\end{document}